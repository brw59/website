\documentclass{beamer}

% for themes, etc.
\mode<presentation>
{ \usetheme{boxes} }

%-- \usepackage{times}  % fonts are up to you
\usepackage{graphicx,ifpdf,hyperref,color,wasysym,colortbl,beamertexpower}
\usepackage{amsmath,alltt,epsfig,xspace,theorem,changebar,textcomp,amssymb}

\usepackage{deletememystyle}

%% Package for drawing figures
\usepackage{tikz}
\usetikzlibrary{arrows}
\usetikzlibrary{calc}
\usetikzlibrary{decorations.markings}
\usetikzlibrary{decorations.pathmorphing}
\usetikzlibrary{decorations.pathreplacing}
\usetikzlibrary{matrix}
\usetikzlibrary{trees}
\usepackage{pgfplots}
\usepackage{subfigure}
\usepackage{skak}

% these will be used later in the title page
\title{Interactive Proof Systems}
\author{CS 480\\[7pt]
    Computational Theory \\[7pt]
    Benjamin Walker}
\date{}

%-----------------------------------------------------------------
\begin{document}

% this prints title, author etc. info from above
\begin{frame}[label=title]
\titlepage
\end{frame}

\begin{frame}[label=pca]
\frametitle{Provers and Verifiers}
Recap: \\
NP is the class of languages that have polynomial time verifiers. \\[10pt]
\pause
\begin{center}
\begin{tabular}{ c c }
{\LARGE Prover} & {\LARGE Verifier} \\
%% \pause
Convince the Verifier & Verify the answer \\
\pause
No computational constraints & Polynomial time only \\

\end{tabular}
\end{center}
\end{frame}

\begin{frame}[label=pca]
\frametitle{Provers and Verifiers}
$SAT \equal (\overline{a} \vee \overline{b} \vee c \vee k \vee \overline{u}) \wedge (a \vee \overline{g}) 
\wedge ... \wedge (r \vee \overline{y} \vee z)$
\\[25pt]
\pause
\begin{center}
\begin{tabular}{ c c }
{\LARGE Prover} & {\LARGE Verifier} \\
%% \pause
Provide Verifier with values & Plug values into SAT problem to verify \\

\end{tabular}
\end{center}
\end{frame}

\begin{frame}[label=pca]
\frametitle{Isomorphic?}
\begin{tikzpicture}
[nodedecorate/.style={shape=circle,inner sep=3pt,draw,thick},%
  linedecorate/.style={-,thick}]
%% nodes or vertices
\foreach \nodename/\x/\y/\direction/\navigate in {2/1/3/above/north,
  1/0/4/above/north, 3/3/2.5/above/north, 4/3/1.5/below/south,
  5/1/1/below/south, 7/-2/2/left/west, 6/0/0/below/south, 8/8/2/above/north, 
  9/0/2/left/west, 10/-2/4/above/north, 11/-2/0/below/south}
{
  \node (\nodename) at (\x,\y) [nodedecorate] {};
  \node [\direction] at (\nodename.\navigate) {$\nodename$};
}
%% edges or lines
\path
\foreach \startnode/\endnode in {1/2, 1/5, 2/3, 2/5, 2/7, 3/4, 3/5,
  4/6, 5/6, 6/7, 10/1, 10/7, 11/7, 11/6, 8/1, 8/6, 8/5, 8/2, 9/2, 9/5}
{
  (\startnode) edge[linedecorate] node {} (\endnode)
};
\end{tikzpicture}
\begin{tikzpicture}
[nodedecorate/.style={shape=circle,inner sep=3pt,draw,thick},%
  linedecorate/.style={-,thick}]
%% nodes or vertices
\foreach \nodename/\x/\y/\direction/\navigate in {2/6/2/above/north,
  1/5/0.8/below/east, 3/3/2/above/north, 4/0.5/1.5/above/north,
  5/8/0/below/south, 7/3/1.5/below/south, 6/1/0.5/below/south, 8/7/1.25/above/north, 
  9/8/1.75/right/east, 10/4/0/below/south, 11/0/1/below/south}
{
  \node (\nodename) at (\x,\y) [nodedecorate] {};
%%  \node [\direction] at (\nodename.\navigate) {$\nodename$};
}
%% edges or lines
\path
\foreach \startnode/\endnode in {1/2, 1/5, 2/3, 2/5, 2/7, 3/4, 3/5,
  4/6, 5/6, 6/7, 10/1, 10/7, 11/7, 11/6, 8/1, 8/6, 8/5, 8/2, 9/2, 9/5}
{
  (\startnode) edge[linedecorate] node {} (\endnode)
};
\end{tikzpicture}
\end{frame}

\begin{frame}[label=pca]
\frametitle{Isomorphic?}
\begin{tikzpicture}
[nodedecorate/.style={shape=circle,inner sep=3pt,draw,thick},%
  linedecorate/.style={-,thick}]
%% nodes or vertices
\foreach \nodename/\x/\y/\direction/\navigate in {2/1/3/above/north,
  1/0/4/above/north, 3/3/2.5/above/north, 4/3/1.5/below/south,
  5/1/1/below/south, 7/-2/2/left/west, 6/0/0/below/south, 8/8/2/above/north, 
  9/0/2/left/west, 10/-2/4/above/north, 11/-2/0/below/south}
{
  \node (\nodename) at (\x,\y) [nodedecorate] {};
  \node [\direction] at (\nodename.\navigate) {$\nodename$};
}
%% edges or lines
\path
\foreach \startnode/\endnode in {1/2, 1/5, 2/3, 2/5, 2/7, 3/4, 3/5,
  4/6, 5/6, 6/7, 10/1, 10/7, 11/7, 11/6, 8/1, 8/6, 8/5, 8/2, 9/2, 9/5}
{
  (\startnode) edge[linedecorate] node {} (\endnode)
};
\end{tikzpicture}
\begin{tikzpicture}
[nodedecorate/.style={shape=circle,inner sep=3pt,draw,thick},%
  linedecorate/.style={-,thick}]
%% nodes or vertices
\foreach \nodename/\x/\y/\direction/\navigate in {2/6/2/above/north,
  1/5/0.8/below/east, 3/3/2/above/north, 4/0.5/1.5/above/north,
  5/8/0/below/south, 7/3/1.5/below/south, 6/1/0.5/below/south, 8/7/1.25/above/north, 
  9/8/1.75/right/east, 10/4/0/below/south, 11/0/1/below/south}
{
  \node (\nodename) at (\x,\y) [nodedecorate] {};
  \node [\direction] at (\nodename.\navigate) {$\nodename$};
}
%% edges or lines
\path
\foreach \startnode/\endnode in {1/2, 1/5, 2/3, 2/5, 2/7, 3/4, 3/5,
  4/6, 5/6, 6/7, 10/1, 10/7, 11/7, 11/6, 8/1, 8/6, 8/5, 8/2, 9/2, 9/5}
{
  (\startnode) edge[linedecorate] node {} (\endnode)
};
\end{tikzpicture}
\end{frame}

\begin{frame}[label=pca]
\frametitle{Provers and Verifiers}
The prover can convince the Verifier of a correct answer in polynomial time, 
\pause
but can the Prover prove to the Verifier that an incorrect answer is not correct in polynomial time? \\[30pt]
\pause
Interestingly, YES! \\[30pt]
\pause
...Provided we give some leeway to our Prover and Verifier definitions. \\
\end{frame}

\begin{frame}[label=pca]
\frametitle{Provers and Verifiers}
\begin{center}
\begin{tabular}{ c c }
{\LARGE Prover} & {\LARGE Verifier} \\
%% \pause
Convince the Verifier & Verify the answer \\
No computational constraints & Polynomial time only \\
\pause
Can engage in a two-way & Allowed to be a Probabilistic \\
dialog with the Verifier & Polynomial Turing machine \\
\end{tabular}
\end{center}
\pause
This is what makes an Interactive Proof System.
\end{frame}

%% \begin{frame}[label=pca]
%% \frametitle{Probabilistic Polynomial Time Machine}
%% \end{frame}

%% This is where I use the board as an example
%%
%% using book example of isomorphic graphs:
%% Verifier randomly reorders nodes of one of the two graphs
%% verifier sends the new graph to the prover, the prover tells the verifier which graph it came from
%% if the prover is consistent many times in a row (arbitrary percentage) then the verifier has been convinced 

\begin{frame}[label=pca]
\frametitle{Isomorphic?}
\begin{tikzpicture}
[nodedecorate/.style={shape=circle,inner sep=3pt,draw,thick},%
  linedecorate/.style={-,thick}]
%% nodes or vertices
\foreach \nodename/\x/\y/\direction/\navigate in {2/1/3/above/north,
  1/0/4/above/north, 3/3/2.5/above/north, 4/3/1.5/below/south,
  5/1/1/below/south, 7/-2/2/left/west, 6/0/0/below/south, 8/8/2/above/north, 9/0/2/left/west, 10/-2/4/above/north, 11/-2/0/below/south}
{
  \node (\nodename) at (\x,\y) [nodedecorate] {};
  \node [\direction] at (\nodename.\navigate) {$\nodename$};
}
%% edges or lines
\path
\foreach \startnode/\endnode in {1/2, 1/5, 2/3, 2/5, 2/7, 3/4, 3/5,
  4/6, 5/6, 6/7, 10/1, 10/7, 11/7, 11/6, 8/1, 8/6, 8/5, 8/2, 9/2, 9/5}
{
  (\startnode) edge[linedecorate] node {} (\endnode)
};
\end{tikzpicture}
\pause
\begin{tikzpicture}
[nodedecorate/.style={shape=circle,inner sep=3pt,draw,thick},%
  linedecorate/.style={-,thick}]
%% nodes or vertices
\foreach \nodename/\x/\y/\direction/\navigate in {2/-2/2/above/north,
  1/0.5/0.5/below/east, 3/4/2/above/north, 4/5/2/above/north,
  5/6/0/below/south, 7/-1.25/1/left/west, 6/1/1/above/north, 
  8/-2/0/below/south, 9/8/1/above/north, 11/-0.75/0.75/left/south, ?/7/1.35/above/north}
{
  \node (\nodename) at (\x,\y) [nodedecorate] {};
%%  \node [\direction] at (\nodename.\navigate) {$\nodename$};
}
%% edges or lines
\path
\foreach \startnode/\endnode in {1/2, 1/5, 2/3, 2/5, 2/7, 3/4, 3/5,
  4/6, 5/6, 6/7, 11/7, 11/6, 8/1, 8/6, 8/5, 8/2, 9/2, 9/5, ?/4, ?/9}
{
  (\startnode) edge[linedecorate] node {} (\endnode)
};
\end{tikzpicture}
\end{frame}

\begin{frame}[label=pca]
\frametitle{Isomorphic?}
\begin{tikzpicture}
[nodedecorate/.style={shape=circle,inner sep=3pt,draw,thick},%
  linedecorate/.style={-,thick}]
%% nodes or vertices
\foreach \nodename/\x/\y/\direction/\navigate in {2/1/3/above/north,
  1/0/4/above/north, 3/3/2.5/above/north, 4/3/1.5/below/south,
  5/1/1/below/south, 7/-2/2/left/west, 6/0/0/below/south, 8/8/2/above/north, 9/0/2/left/west, 10/-2/4/above/north, 11/-2/0/below/south}
{
  \node (\nodename) at (\x,\y) [nodedecorate] {};
  \node [\direction] at (\nodename.\navigate) {$\nodename$};
}
%% edges or lines
\path
\foreach \startnode/\endnode in {1/2, 1/5, 2/3, 2/5, 2/7, 3/4, 3/5,
  4/6, 5/6, 6/7, 10/1, 10/7, 11/7, 11/6, 8/1, 8/6, 8/5, 8/2, 9/2, 9/5}
{
  (\startnode) edge[linedecorate] node {} (\endnode)
};
\end{tikzpicture}
\begin{tikzpicture}
[nodedecorate/.style={shape=circle,inner sep=3pt,draw,thick},%
  linedecorate/.style={-,thick}]
%% nodes or vertices
\foreach \nodename/\x/\y/\direction/\navigate in {2/-2/2/above/north,
  1/0.5/0.5/below/east, 3/4/2/above/north, 4/5/2/above/north,
  5/6/0/below/south, 7/-1.25/1/left/west, 6/1/1/above/north, 
  8/-2/0/below/south, 9/8/1/above/north, 11/-0.75/0.75/left/south, ?/7/1.35/above/north}
{
%% {2/2/2/above/north,
%%  1/5/0/below/south, 3/8/0/above/north, 4/0.25/0.75/below/east,
%%  5/7/1.5/below/south, 7/3/1/below/south, 6/-2/0.90/above/north, 
%% 8/2/0/below/south, 9/.75/1.75/left/north, 11/8/2/below/south, ?/-.2/1.35/above/north}
  \node (\nodename) at (\x,\y) [nodedecorate] {};
  \node [\direction] at (\nodename.\navigate) {$\nodename$};
}
%% edges or lines
\path
\foreach \startnode/\endnode in {1/2, 1/5, 2/3, 2/5, 2/7, 3/4, 3/5,
  4/6, 5/6, 6/7, 11/7, 11/6, 8/1, 8/6, 8/5, 8/2, 9/2, 9/5, ?/4, ?/9}
{
  (\startnode) edge[linedecorate] node {} (\endnode)
};
\end{tikzpicture}
\end{frame}

\begin{frame}
\frametitle{Uses of Interactive Proof Systems}
\pause
\begin{enumerate}
\item Profoundly affected complexity theory [Multiple Provers]
%% Example of such is of an interrogation. If both convicted felons (the provers) give the same story as they are being interrogated separately by the verifier, then it is probably that they are telling the truth.
\pause
\item Advances in cryptography [Zero Knowledge]
%% https://en.wikipedia.org/wiki/Zero-knowledge_proof
%% talk about the hamilton circuit problem
%%
%% Given the Prover claims there is a Hamilton Circuit in a large graph G (which is computationally unfeasible)
%% Prover makes another graph H that he claims is isomorphic
%% Verifier can ask either "Show me that Graph H is isomorphic to Graph G" or "Show me the Hamilton Circuit in Graph H". The Verifier can only ask one of those questions for every Graph H that the Prover generated.
%%
%% After so many graphs, it becomes very probable that the prover was telling the truth and the prover never gave away the Hamilton Circuit in graph G.
\pause
\item Advances in approximation algorithms
%% Approximation algorithms are typically used to find approximate solutions to optimization problems; they are typically associated with NP-Hard problems.
%%
%% https://en.wikipedia.org/wiki/Vertex_cover
%% minimum vertex cover ... approx algorithm guarantees a solution (not necessarily minimum) with a ceiling of up to 2 times the size of the minimum.

%% Another example is the Lattice problem
\end{enumerate}
\end{frame}

\begin{frame}
\frametitle{Things I didn't talk about}
\begin{enumerate}
\item Approximate Shortest Lattice Vector is another one of the "elusive" problems
\begin{enumerate}
\item "elusive" = NP Problems not known to be in P or to be NP-Complete
\item Approximation algorithm techniques (Interactive Proof System techniques) are used to help find answers to this problem
\end{enumerate}
\item The set of languages which have interactive proof systems is equivalent to PSPACE
\item MIP (Multiprover Interactive Proofs) is equivalent to NEXP
\end{enumerate}
\end{frame}

\end{document}

%% Sources:
%% http://www.cs.princeton.edu/courses/archive/spr06/cos522/ip.pdf
%% https://en.wikipedia.org/wiki/Lattice_problem
%% https://en.wikipedia.org/wiki/Vertex_cover
%% https://en.wikipedia.org/wiki/Zero-knowledge_proof
%% https://www.iacr.org/workshops/tcc2007/Micciancio.pdf
